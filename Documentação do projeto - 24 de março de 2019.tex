\documentclass{article}
\usepackage[utf8]{inputenc}

\title{\textbf{Documentação do Projeto}}
\author{\textbf{.PNGenética}}
\date{24 de Março de 2019}

\begin{document}
	
	\maketitle
	
	\section{Nome do Projeto, contexto e objetivos}
	\begin{itemize}
		\item {\textbf{Projeto}}: .PNGenética \newline
	\end{itemize}

	Em um cenário de extrapolação dos meios de produção artística, torna-se latente o questionamento quanto aos limites da mesma: o que é arte ? O que a rege ? Qual o requisito a sua produção ? Como definir o que é belo ? Evidentemente algumas dessas perguntas continuam a direcionar a discussão em torno da conceituação do indivíduo, de sua mente e subconsciente de modo a fazer com que a humanidade aproxime-se o máximo do possível de uma resposta. Nesse sentido, em um contexto de desenvolvimento técnico exponencial, arte e tecnologia caminham juntas recriando as formas pelas quais as duas relacionam-se de modo a promover o contato com o indivíduo. \newline

	Sendo assim, em conjunto com o grupo "Amudi", segundo a orientação e direcionamento do professor Edson S. Gomi e do monitor João Flesch Fortes, nós, membros do projeto ".PNGenética", decidimos contribuir para o enriquecimento da discussão em torno da relação entre arte e tecnologia. Com esse intuito, fazendo uso de algoritmos genéticos e da plataforma Raspberry Pi, iremos desenvolver ao longo do primeiro semestre de 2019, a criação de imagens, inicialmente abstratas, e disponibilizá-las em site onde as pessoas poderão acessar e votar nas mais interessantes. A partir dessa seleção, as mais "aptas" segundo a ótica dos votantes seguirão para reprodução e mutação, criando, deste modo, uma nova geração que retornará ao processo numa nova iteração.\newline

	O objetivo do projeto ao produzir imagens de acordo com a preferência do inconsciente coletivo é entender melhor quais os estímulos visuais são responsáveis pelo prazer derivado da arte e como se dá a relação entre ser humano e a mesma. Finalmente, tendo sido feita esta análise, as aplicações do projeto podem vir a permear desde os setores produtivos até a educação de modo a promover entre cultura e sociedade ainda na fase de desenvolvimento da criança e do adolescente. 

	\section{Requisitos do projeto}
	Conforme estipula na proposta do projeto de PCS3100, os requisitos são:
	\begin{itemize}
		\item Utilizar o Raspberry Pi 3.
		\item Produzir novas imagens a partir do feedback do público.
	\end{itemize}
	
	\section{Especificação do projeto}
	O ".PNGenética" será um projeto que produzirá periodicamente imagens inéditas com base na seleção a partir de votações públicas. Ele funcionará segundo a seguinte sequência:
	\begin{itemize}
	\item Produzir imagens segundo um código aleatório.
	\item Publicar as imagens e fazer uma votação pública para selecionar qual é o grau de aceitação de cada uma segundo as preferências do público.
	\item Gerar uma nova geração por meio do algoritmo genético com a informação da preferência do público. 
	\item Repetir o processo de votação e do algoritmo genético
	\end{itemize}

	\section{Materiais e Métodos}
	Os produtos, serviços e sistemas utilizados segundo a concepção inicial do projeto pelo grupo serão: 
	\begin{itemize}
		\item Raspberry Pi 3.
		\item Linguagem Python para a codificação do gerador de imagens.
		\item Servidor externo para a manutenção de um site onde os resultados serão apresentados.
		\item Desenvolvimento de métodos de maximização do alcance do número de votantes para o experimento.
	\end{itemize}

	\section{Testes}	
	Para garantir o funcionamento do projeto, utilizaremos de alguns testes:
	\begin{itemize}
		\item Verificar o funcionamento do servidor do site antes das votações.
		\item Introduzir uma versão beta das votações para um grupo seleto de candidatos pré-selecionados.
		\item Analisar os retornos obtidos e dar continuidade ao processo.
	\end{itemize}
	
	\section{Resultados e Conclusão}
	Nesta seção da documentação do projeto serão disponibilizados as implicações finais do projeto. Como no dia 24 de março o mesmo ainda encontra-se na fase embrionária, este tópico será usado posteriormente.
	
	\section{Integrantes}
	\begin{itemize}
		\item Bernardo Coutinho
		\item Joás Barbosa
		\item Lucas Cupertino
		\item Stephanie Miho
	\end{itemize}

	
\end{document}